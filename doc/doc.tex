%! Author = Patrik Skaloš


% Preamble
\documentclass[a4paper]{article}


% Packages
% \usepackage[utf8]{inputenc}
\usepackage[czech]{babel}
% \usepackage[T1]{fontenc}
\usepackage[total={18cm, 25cm}]{geometry}
\usepackage{amsmath, amssymb, amsthm}
\usepackage{hyperref}
% \usepackage{graphicx}

% Title
\title{Tunelovanie dátových prenosov protokolom DNS}
\author{Patrik Skaloš}
% \date{}


% Document
\begin{document}

  \renewcommand{\contentsname}{Obsah}

  \maketitle

  \tableofcontents



  \section{Úvod}

  \subsection{Protokol DNS}

  DNS (\emph{Domain Name System}) je protokol používaný na preklad doménového
  mena (napr. \emph{www.example.com}) na IP adresu použiteľnú na prístup k
  obsahu uloženého pod touto doménou. Klient, ak chce k tomtu obsahu pristúpiť,
  musí najprv poslať požiadavku (\emph{query}) DNS serveru, ktorý na požiadavku
  odpovie IP adresou, ak má v svojej databázi záznam so žiadaným doménovým
  menom. V opačnom prípade je potrebné požiadavku poslať DNS serveru vyššej
  úrovne.


  \subsection{Tunelovanie dát cez DNS}

  Tunelovanie dát protokolom DNS (ďalej DNS tunelovanie) je metóda enkapsulácie
  dát do DNS datagramu (paketu). 

  Klientské zariadenie môže mať ten najprísnejší firewall, no ak má
  povolené datagramy (pakety) DNS, využitím DNS tunelovania je možné medzi ním
  a serverom vytvoriť komunikačný kanál, keďže bude komunikácia vedená čisto cez
  požiadavaky a odpovede podľa protokolu DNS.

  Príklad otázky v DNS požiadavke obsahujúcej nezakódované dáta - užívateľské
  meno \emph{username} a heslo \emph{password}:
  \textbf{username.password.example.com}

  \textbf{Poznámka}: Podľa štandardu
  \emph{RFC1035}\footnote{\href{https://www.ietf.org/rfc/rfc1035.txt}{RFC1035:
  https://www.ietf.org/rfc/rfc1035.txt}} musí doménový štítok (\emph{angl.
    label} - časť domény ohraničená bodkou, začiatkom alebo koncom domény)
    začínať písmenom, končiť písmenom alebo číslom a obsahovať iba písmená,
    čísla a pomlčku. To je jeden z dôvodov, prečo sa dáta enkapsulované v URL
    požiadavke enkódujú do formátu obsahujúceho iba povolené znaky.


  \subsection{Prečo DNS tunelovanie funguje}

  Ak chce klient používať protokol DNS, nemôže ho, samozrejme, zakázať. Jedinou
  ochranou pred DNS tunelovaním potom ostáva rozbaľovať DNS datagramy (pakety)
  a filtrovať ich podľa požiadavky. Môže však byť veľmi náročné určiť, či je
  požiadavka dôveryhodná alebo nie.



  \section{Impementácia}

  \subsection{Klient}

  Klient (po spracovaní argumentov) prečíta a uloží všetky vstupné dáta zo
  štandardného vstupu alebo súboru, podľa argumentov spustenia. Tieto dáta
  následne zakóduje do formátu \textit{base64} a je pripravený na ich
  odoslanie.

  Na prenos dát klient používa iba protokol \textit{UDP} (User Datagram
  Protocol), ktorý je známy svojou nespoľahlivosťou. V tomto protokole
  odosielateľ jednoducho pošle svoje dáta, pričom ich doručenie nie je zaistené
  a o doručení odosielateľ žiadne potvrdenie nedostane. Keďže je pre nás
  dôležité, aby serveru dáta prišli v poriadku (a všetky), bolo potrebné
  implementovať lepší mechanizmus prenosu dát. Tento mechanizmus je popísaný v
  podsekcii \hyperref[label]{Komunikácia: Prenos dát medzi klientom a
  serverom}.

  \textbf{Poznámka}: Formát Base64 je sada 64 znakov (ktoré sa vojdú do 6
  bitov) určená hlavne na zakódovanie dát, ktoré by mali obsahovať iba
  tlačiteľné znaky (\textit{angl. printable characters}). Pozostáva z 26 malých
  a 26 veľkých písmen abecedy, desiatich cifier a znakov \verb|+| a \verb|/|.


  \subsection{Server}
  Server po spracovaní argumentov začína naslúchať na porte 53 (predvolený port
  pre DNS komunikáciu) a kým nie je program zastavený, v slučke prijíma
  datagramy od klienta. Po prijatí všetkých dát (zakódovaných do formátu
  \textit{base64}) ich dekóduje a zapíše do výstupného súboru, ktorý je tvorený
  povinným parametrom \textit{DST\_FILEPATH} (ktorý značí priečinok, do ktorého
  sa dáta uložia) a relatívnou cestou prijatou od klienta. Proces komunikácie
  je podrobnejšie popísaný v nasledujúcej sekcii.


  \subsection{Komunikácia}

  \subsubsection{Enkódovanie dát do DNS otázky}

  Podľa \textit{RFC1035}, doména v DNS otázke môže obsahovať až štyri štítky, z
  ktorých dva posledné štítky (doménové meno a doména najvyššej úrovne -
  \textit{angl. extension}) sú definované parametrom (klienta aj serveru)
  \textit{BASE\_HOST}.  Ostávajú nám teda dva štítky pre naše dáta, pričom
  ďalej \textit{RFC1035} stanovuje, že štítok môže mať maximálnu dĺžku 63 bajtov.
  Do jedného DNS datagramu nám teda vojde 126 bajtov. Keďže dáta kódujeme do
  formátu \textit{base64}, ktorý dáta nafúkne na $\frac{4}{3}$ pôvodnej
  veľkosti, do datagramu sa vojde približne 94 bajtov užívateľkých dát.

  Dva príklady enkódovania dát do \textit{base64} a do doménového mena pri
  parametri \textit{BASE\_HOST = xskalo01.com}:
  \begin{itemize}
    \item 
      \begin{itemize}
        \item Uživateľské dáta: \verb|Ahoj|
        \item Base64: \verb|QWhvag|
        \item Doména v DNS otázke: \verb|QWhvag.xskalo01.com|
      \end{itemize}
    \item 
      \begin{itemize}
        \item Uživateľské dáta: \verb|01234567890123456789012345678901234567890123456789|
        \item Base64: \verb|MDEyMzQ1Njc4OTAxMjM0NTY3ODkwMTIzNDU2Nzg5MDEyMzQ1Njc4OTAxMjM0NTY3ODk|
        \item Doména v DNS otázke: \verb|MDEyMzQ1Njc4OTAxMjM0NTY3ODkwMTIzNDU2Nzg5MDEyMzQ1Njc4OTAxMjM0NTY| \\ 
          \verb|.3ODk.xskalo01.com|
      \end{itemize}
  \end{itemize}


  \subsubsection{Prenos dát medzi klientom a serverom}
  \label{label}

  Do prvej správy jednej komunikácie sa vždy zakóduje iba relatívna cesta
  udávajúca, kam má server uložiť dáta, ktoré nasledujú. Predpokladáme, teda,
  že táto cesta (zakódovaná) bude kratšia ako 126 bajtov, čo v niektorých
  prípadoch nemusí byť dosť, no nám to postačí.

  Do nasledujúcich datagramov sa vkladajú zakódované dáta - vždy maximálne 126
  bajtov (znakov). Server tieto dáta postupne ukladá do RAM a dekóduje ich až
  na konci komunikácie.

  Posledná správa, značiaca koniec komunikácie, neobsahuje žiadne zo
  zakódovaných dát. Doména v DNS otázke sa teda rovná argumentu
  \textit{BASE\_HOST}. Tento spôsob nie je efektívny a zvolili sme si ho iba
  kvôli jednoduchosti implementácie.

  Ako sme už naznačili, UDP nie je spoľahlivý komunikačný protokol. Z toho
  dôvodu server na každú prijatú správu klientovi odpovedá za účelom potvrdenia
  prijatia rovnakým datagramom, ako prijal, s jedinou zmenou - v hlavičke DNS
  požiadavke nastaví príznak \textit{odpoveď} (\textit{angl. response}). Časový
  limit na obdržanie potvrdenia o prijatí je nastavený na 100 ms. V prípade, že
  klient neobdrží potvrdenie o prijatí, komunikácia musí byť ukončená a začatá
  odznovu. Aby sa klient nezasekol pri odosielaní dát nenaslúchajúcemu serveru,
  použili sme premennú \textit{MAX\_TRIES}, ktorá je štandardne nastavená na 3
  a definuje počet pokusov na odosielanie jedného súboru a zároveň počet
  pokusov na odoslanie datagramu ukončujúceho komunikáciu.

  \begin{samepage}
    Poradie komunikácie v prípade, že klientovi nie je doručené potvrdenie o prijatí:
  \begin{enumerate}
    \item Klient odošle datagram serveru a nedostane odpoveď
    \item Klient sa pokúsi predčasne uzavrieť komunikáciu odoslaním požiadavky
      bez dát (ukončujúca požiadavka)
    \item Ak sa uzavretie komunikácie nepodarilo (server nepotvrdil prijatie,
      resp. potvrdenie nebolo klientovi doručené) a o uzavretie komunikácie sa
      program pokúsil menej ako \textit{MAX\_TRIES} krát, pokračuje krokom 2.
    \item Ak sa uzavretie komunikácie nepodarilo ani na posledný pokus, kanál
      medzi klientom a serverom je nefunkčný, prenos dát zlyhal a program
      klienta je ukončený s chybovou hláškou.
    \item Ak sa predčasné uzavretie komunikácie podarilo, môžeme sa znovu
      pokúsiť odoslať dáta serveru. Začína sa znovu, od prvej požiadavky
      obsahujúcej cieľovú cestu. V prípade, že prenos dát zlyhal už
      \textit{MAX\_TRIES} krát, program klienta je ukončený s chybovou hláškou.
  \end{enumerate}
  \end{samepage}
      

  \subsection{Nedostatky, obmedzenia a rozšírenia}
  Nasleduje neúplný zoznam nedostatkov, obmedzení a možných rozšírení našej
  implementácie (väčšina z nich vyplýva zo zamerania sa na jednoduchosť):
  \begin{itemize}
    \item Kódovanie dát do formátu \textit{base64} znamená, že (len v prípade
      kódovania binárnych dát) sa v DNS otázke môžu vyskytovať znaky \verb|+| a
      \verb|/|, ktoré však podľa \textit{RFC1035} v doméne nie sú povolené.
      Takéto DNS požiadavky sme pri testovaní odosielali aj na bežné DNS
      servery, ktoré s tým, kupodivu, nemali problém. Z toho dôvodu sme
      implementáciu neprarábali a naďalej kódujeme do \textit{base64}, no
      určite by bolo dobrým nápadom kódovať do formátu, ktorý obsahuje iba
      písmená (veľké aj malé), čo by bez problémov vyhovovalo štandardu. Našim
      nápadom bolo implementovať vlastný \textit{base48}, ktorý by dáta
      enkódoval len na malé a veľké písmená abecedy. Riešilo by to viac
      problémov nášho projektu s \textit{RFC1035}
    \item Dáta sú síze zakódované, no nie sú zašifrované! Ktokoľvek, kto odchytí
      naše datagramy by dokázal vcelku ľahko dekódovať dáta, ktoré odosielame.
    \item Odhalenie toho, že datagramy, ktoré posielame, nie sú reálne DNS
      požiadavky by bolo vcelku jednoduché, keďže v každej DNS otázke využívame
      priestor do plnej miery (maximálnych 126 bajtov v doméne) a potvrdenia o
      prijatí sú nezmyselné.
    \item Prenos veľkých súborov, alebo aj menších súborov po menej stabilnej
      môže ľahko zlyhať z dôvodu použitia UDP protokolu a iba jednoduchého
      mechanizmu zaistenia prenosu.
    \item Prenos dát je z dôvodu veľkej réžie neefektívny.
    \item Dĺžka reťazca označujúci relatívnu cestu pre uloženie súboru môže byť
      dlhá maximálne (približne) 94 bajtov (znakov).
    \item Celý odosielaný súbor či prijaté dáta sú uložené do RAM, preto naša
      implementácia nie je vhodná pre prenos veľkých súborov.
    \item \ldots
  \end{itemize}



  \section{Testovanie}

  Testovanie programov klienta a serveru prebiehali manuálne aj automaticky,
  na našom zariadení s operačným systémom \textit{Debian 11} aj na školskom
  serveri s operačným systémom \textit{FreeBSD 13}.


  \subsection{Manuálne testovanie}

  Programy klienta aj serveru boli manuálne otestované s rôznymi argumentmi a 
  vstupmi s rôznou dĺžkou aj typom (binárne aj textové dáta).


  \subsection{Testovací skript}

  Na to, aby sme otestovali prenos čo najviac rôznych kombinácií dát, sme
  vytvorili jednoduchý skript v jazyku \textit{python}. Skript generoval
  náhodné ASCII znaky náhodnej dĺžky, zapísal ich do súboru a spustil program
  klienta. Po konci behu programu skontroluje, či je vstupný súbor identický
  so súborom vytvoreným serverom.



\end{document}
